مسئولیت تمرین‌های درس با
\href{https://t.me/AliRZ888}{علی رحمی‌زاد}
است. در صورت بروز هرگونه مشکل در رابطه با تمارین، به ایشان مراجعه کنید.
 
\section{قوانین حل تمرین}
\begin{itemize}[itemsep=1em, parsep=0.5em]
    \setstretch{1.25}
    \item
    تمرین‌های درس چهارشنبه‌ی هر هفته، راس ساعت ١٨:۲۳ از طریق کانال تلگرام درس در دسترس خواهند بود. دانشجویان تنها از بین سوالات غیرتحویلی مشخص‌شده، می‌توانند سوال مورد نظرشان را انتخاب کرده و نهایتاً تا ساعت ٢٣ همان شب در گروه تلگرام درس با هشتگ حل تمرین و به فرمت زیر آمادگی خود را برای حل سوال اعلام کنند:
    
    ١ . شماره سوال مورد نظر ‐ نام سوال مورد نظر
    \\
    ٢ . نام و نام خانوادگی خود
    \\
    ٣ . تعداد دفعاتی که سوال حل کرده‌اید (بار چندم است که در تمرین‌ها سوال حل می‌کنید؟)
     
    برای مثال:

    \[\text{\textcolor{darkgreen}{\#حل تمرین}}\]
    \[\text{\textcolor{darkgreen}{سوال شماره ۱ ‐ تدوین قوانین حل تمرین}}\]
    \[\text{\textcolor{darkgreen}{پارمیدا جوادیان}}\]
    \[\text{\textcolor{darkgreen}{بار دوم}}\]
    
    \item
    جلوی هر سوال، نام دستیار آموزشی مربوطه نوشته شده است. پیش از مراجعه به دستیار آموزشی مربوطه، لازم است آمادگی کافی برای حل سوال را داشته باشید. در صورت سخت بودن سوال انتخابی خود، می‌توانید با دستیار آموزشی مربوطه مشورت کنید تا منابع لازم را به شما معرفی کند. توجه کنید که حل کردن سوال بر عهده‌ی شماست و دستیار آموزشی تنها شما را در راستای حل سوال راهنمایی خواهد کرد.

    \item
    حتما ابتدا راه حل خود و جزئیات آن را با دستیار آموزشی مربوطه بررسی کنید و سپس اقدام به ساخت ویدئو کنید. در نهایت علاوه بر راه حل شما، ویدئوی ارسالی نیز باید مورد تایید دستیار آموزشی مربوطه قرار بگیرد.
    \textbf{توجه کنید که راه حل نهایی شما باید درست باشد.}
    حتماً نکات مطرح شده در بخش دوم قوانین را در تهیه‌ی ویدئوی خود رعایت کنید.

    \item
    حداکثر تا اولین یک‌شنبه‌ی پس از انتشار تمرین فرصت دارید تا ویدئوی حل نهایی خود را برای دستیار آموزشی مربوطه ارسال نمایید.

    \item
    نمرهٔ حل هر سوال رابطهٔ معکوس با تعداد دفعات حل سوال توسط شما دارد. به طوری که برای بار اول
    $0.4$،
     بار دوم
    $0.3$،
    و بار سوم
    $0.2$
    نمره به شما تعلق خواهد گرفت.

    \item
    در صورتی که برای حل سوالی داوطلب شوید اما در ادامه حل نهایی آن را به دستیار آموزشی مربوطه نفرستید، نمره‌ی منفی دریافت خواهید کرد.

    \item
    اولویت حل سوال، در مرتبه‌ی اول با فردی است که سوالات کمتری حل کرده باشد. در مرحله‌ی دوم اولویت با فردی است که زودتر آمادگی خود را اعلام کرده باشد.

    \item
    ویدئوهای شما در سامانه‌ی cw درس بارگذاری خواهد شد.

    \item
    برای هر تمرین یک کلاس حل تمرین برگزار می‌شود. زمان برگزاری این کلاس ساعت ۱۶ چهارشنبه‌ی هفته‌ی بعد از انتشار تمرین است. حضور افرادی که برای یک تمرین ویدئو تهیه کرده‌اند، در کلاس حل تمرین مربوطه الزامی است.

\end{itemize}

\section{نکات تهیه‌ی ویدئو}

\begin{itemize}[itemsep=1em, parsep=0.5em]
    \setstretch{1.25}
    \item
    ویدئوی تهیه شده توسط شما باید از کیفیت لازم برخوردار باشد و راه‌حل را به طور کامل به دانشجویان منتقل کند. از تهیه‌ی ویدئوهای بیش از حد طولانی و یا نامفهوم جداً خودداری کنید. همچنین برای تهیه‌ی ویدئو حتماً از ابزار مناسب استفاده کنید.

    \item
    برای تهیهٔ ویدئو، باید تصویر خودتان را با استفاده از ابزارهایی مانند
    \href{https://www.bandicam.com/downloads/}{\lr{bandicam}}
    در گوشه‌ای از ویدئوی ضبط شده قرار دهید.

    \item
    در ابتدای ویدئو خودتان را معرفی کنید. همچنین صورت سوال را دقیق و واضح توضیح دهید. استفاده از یک مثال ساده و کوچک برای شفاف‌تر شدن صورت سوال توصیه می‌شود.

    \item
    دقت کنید کیفیت صدا و تصویر ویدئو مناسب باشد. همچنین سرعت حرف زدنتان آهسته و یا تند نباشد.

    \item
    در صورتی که در راه حل خود از قضیه یا هر نکته‌ای که سر کلاس آموزش داده شده است استفاده می‌کنید، آن را بیان نمایید.

    \item
    مدت زمان ویدئوی ارسالی شما باید ۵ دقیقه باشد. در غیر این صورت با دستیار آموزشی مربوطه هماهنگ کنید.

    \item
    در صورتی که قصد ساخت اسلاید برای توضیحات خود را دارید، دقت کنید که اسلایدها پر از نوشته نباشند؛ چرا که دانشجویان به توضیحات شما دقت می‌کنند و نمی‌توانند همزمان با توضیحات شما اسلایدهای شلوغ را نیز دنبال کنند.

    \item
    سعی کنید با استفاده از شکل، فهم راه حل خود را آسان تر کنید. در صورتی که سوال شما قابلیت توضیح با یک مثال را دارد، حتماً مثال بزنید.

    \item
    می‌توانید برای توضیح راه حل خود، آن را روی کاغذ بنویسید و با موبایل از آن فیلم بگیرید. همچنین می‌توانید از ابزارهایی که تخته‌ی مجازی در اختیار شما قرار می‌دهند، مانند
    \href{https://miro.com/online-whiteboard/?awwapp_ref=direct&utm_source=awwapp&utm_campaign=direct&utm_name=awwapp_redirect}{\lr{awwapp}}
    استفاده کنید. در هر صورت دقت کنید که نوشته‌های شما خوانا بوده و ریز نباشند.

\end{itemize}

\begin{flushleft}
    \textbf{موفق باشید}
\end{flushleft}