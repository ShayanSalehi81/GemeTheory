دو کشور
A
و
B
را در نظر بگیرید که بر سر یک منبع ذخایر نفتی ارزشمند به ارزش ۲۰ میلیارد دلار با هم اختلاف دارند. هر دو کشور به دنبال کنترل انحصاری این منبع هستند و برای حمایت از ادعای خود از کشور ثالثی به نام
C
کمک می گیرند. هر کدام از این کشورها می توانند برای تحت تاثیر قرار دادن تصمیم کشور
C
به آن رشوه دهند. مبلغ رشوه می تواند تنها ۹ میلیارد دلار  یا ۲۰ میلیارد دلار باشد. هر مبلغ دیگری باعث رد صلاحیت می شود. کشوری که بالاترین رشوه را پیشنهاد دهد به عنوان مالک قانونی منبع شناخته خواهد شد. اگر هر دو کشور مبلغ یکسانی را رشوه دهند، احتمال دریافت منبع برای هر کدام مساوی است. همچنین اگر هیچ یک از کشورها رشوه ندهند، احتمال دریافت منبع برای هر کدام به طور مساوی می باشد.
\vspace{10pt}

\textbf{الف)}
تعادل نش خالص این بازی را پیدا کنید.
\vspace{5pt}

\textbf{ب)}
بخش قبل را برای حالتی که امکان رشوه 15 میلیاردی وجود داشته باشد، حل کنید.
\vspace{5pt}

\textbf{ج)}
تعادل نش ترکیبی را برای حالت جدید بدست آورید.