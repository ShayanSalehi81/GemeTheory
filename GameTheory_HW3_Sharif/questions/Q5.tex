شما و دوستتان در حال انجام یک بازی ویدیویی هستید. هر دوی شما معتقدید که احتمال
$\pi$
وجود دارد که قلمرو مشخصی در حال حاضر توسط حریفان قدرتمند اشغال شده باشد. اگر این قلمرو واقعاً اشغال شده باشد، هر بازیکنی که سعی کند آن را فتح کند، مطمئناً با شکست مواجه خواهد شد. در صورتی که شکست بخورید امتیاز
$-c$
و در صورت فتح آن امتیاز 1 بدست خواهید آورد. در صورت تصمیم بر عدم حمله امتیازی کسب نخواهید کرد. دقت کنید که فقط دو بار برای حمله به قلمروی دشمن فرصت دارید. اگر یکی از شما در فرصت اول در فتح این قلمرو شکست بخورد، هر دوی شما نتیجه می گیرید که تلاش برای تصرف آن منجر به شکست خواهد شد و بنابراین از تلاش برای تصرف در فرصت بعد خودداری می‌کنید. اگر هیچ یک از شما در فرصت اول شکست نخورید، هر دو به این باور ادامه می دهید که احتمال شکست برابر با
$\pi$
است و بنابراین تنها در صورتی در فرصت دوم حمله می‌کنید که:
\[
-\pi c + 1 - \pi \geq 0  
\]

\textbf{الف)}
این وضعیت را به عنوان یک بازی استراتژیک مدل کنید که در آن هر کدام از شما تصمیم می گیرید که در فرصت اول بازی حمله کنید یا خیر.
\vspace{5pt}

\textbf{ب)}
تعادل های ترکیبی این بازی را بر اساس پارامترهای
$c$
و
$\pi$
شناسایی کنید. علاوه بر این، تحلیل کنید که آیا حضور یک دوست بر تصمیم شما برای تصرف سرزمین ها در روز اول تأثیر می گذارد یا خیر.