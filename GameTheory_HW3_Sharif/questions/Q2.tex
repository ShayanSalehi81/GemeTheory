در یک شهر، انتخابات شهرداری در حال برگزاری است و دو کاندیدا به نام های آلیس و باب برای کسب آرای دو منطقه با جمعیت و الگوی رأی گیری متفاوت رقابت می کنند. آلیس سه گروه تبلیغاتی دارد، در حالی که باب دو گروه دارد. هر کاندیدا می تواند گروه های تبلیغاتی خود را بین دو منطقه تقسیم کند. آلیس در صورتی انتخابات هر منطقه را برنده می شود که گروه های تبلیغاتی بیشتری نسبت به باب به آن منطقه اختصاص دهد. تنها در صورتی آلیس برنده انتخابات کل شهر می شود که در هردو منطقه برنده انتخابات شود.
\vspace{10pt}

\textbf{الف)}
این موقعیت را به عنوان یک بازی استراتژیک مدل سازی کنید و تمامی تعادل های استراتژی ترکیبی را پیدا کنید.
\vspace{5pt}

\textbf{ب)}
در یک تعادل، آیا کاندیداها تمام تلاش خود را روی یک منطقه متمرکز می کنند یا گروه های تبلیغاتی خود را تقسیم می کنند.