استانداری فارس قصد دارد بین شیراز و نورآباد یک جاده بسازد. هزینه ساخت جاده برابر با
$0 < c$
فرض می‌شود. ارزش این جاده برای شهرداری شیراز برابر با
$0 \leq v_S$
و برای شهرداری نورآباد برابر با
$0 \leq v_N$
فرض می‌شود.
\vspace{2pt}

هر کدام از شهرداری‌های شیراز و نورآباد به‌طور همزمان و به ترتیب مبالغ
$0 \leq b_S$
و
$0 \leq b_N$
را برای مشارکت در ساخت جاده پیشنهاد می‌کنند. اگر
$c \leq b_S + b_N$
باشد، جاده ساخته می‌شود. اگر
$b_N < c \leq b_N + b_S$،
شهرداری شیراز مبلغ
$c - b_N$
را به استانداری پرداخت می‌کند؛ در غیر این صورت هم مبلغی را پرداخت نمی‌کند. به طور مشابه اگر
$b_s < c \leq b_N + b_S$
شهرداری نورآباد مبلغ
$c - b_s$
را به استانداری پرداخت می‌کند. در غیر این صورت هم مبلغی را پرداخت نمی‌کند.
\vspace{2pt}

اگر جاده ساخته شود، سود هر شهرداری برابر با ارزش جاده برای آن منهای میزان پرداختی به استانداری است. در غیر این صورت هم برابر با صفر است. با در نظر گرفتن فرض
$v_S + v_N < c$
به سوالات زیر جواب دهید:

\vspace{10pt}

\textbf{الف)}
این بازی را به فرم نرمال بنویسید.
\vspace{5pt}

\textbf{ب)}
بررسی کنید که آیا تعادل استراتژی غالب وجود دارد؟ در صورت وجود آن را محاسبه کنید.