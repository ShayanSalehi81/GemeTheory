فرض کنید موسوی و کوروش دو فروشنده گوشی موبایل هستند. هر دو می‌خواهند با تعیین قیمت مناسب سود خود را از بازاری که برای فروش محصولشان وجود دارد بیشینه کنند.
\vspace{2pt}

فرض کنید کل هزینه‌ای که هر شرکت برای تهیه‌ی هر گوشی موبایل انجام می‌دهد ۸ دلار باشد. اگر قیمتی که موسوی تعیین می‌کند
$P_M$
و قیمتی که کوروش تعیین می‌کند
$P_K$
باشد، میزان فروش موسوی برابر با:
$$Q_M = 44 - 2 P_M + P_K$$
و میزان فروش کوروش برابر با:
$$Q_k = 44 - 2 P_K + P_M$$
خواهد بود.
\vspace{2pt}

به کمک تعیین
\LR{best response} 
ها تعادل نش را در میان مجموعه‌ی تصمیمات این دو محاسبه کنید. استفاده از نمودار و نشان دادن نتایجتان به شکل تصویری بسیار توصیه می‌شود.
