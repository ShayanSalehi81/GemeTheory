مسئله‌ی مزایده‌ی آنلاین تک کالا، به این صورت فرمول‌بندی می‌شود. شما می‌خواهید یک کالا را به فروش بگذارید و $n$ خریدار در بازه‌های زمانی
$t = 1, \ldots , n$
به نوبت پیشنهاد خودشان را اعلام می‌کنند. شما به عنوان طراح این مزاید‌ه‌ی آنلاین، در هر بازه‌ی زمانی i باید تصمیم بگیرید که اولا کالایتان را به خریدار $i$ام بفروشید و یا دست نگه دارید تا پیشنهاد نفر بعدی را بدانید و طبیعتا خریدار $i$ام را از دست می‌دهید. ثانیا باید تعیین کنید که با چه قیمتی کالایتان را می‌فروشید. مکانیزم زیر را در نظر بگیرید:
\begin{itemize}
    \item
    در هر بازه‌ی زمانی $i$، که تا آن زمان کالا به فروش نرفته است، یک قیمت $p_i$ توسط شما و به گونه‌ای که خریداران نمی‌دانند تعیین می‌شود و سپس پیشنهاد خریدار $i$ام ($b_i$) را بشنوید. (یعنی قیمتگذاری شما تابع $b_i$ نیست). اگر $b_i \geq p_i$ آنگاه کالا را به خریدار iام می‌فروشید. و گرنه کالا برای بازه‌ی زمانی بعد فروخته نشده می‌ماند.
\end{itemize}

\vspace*{5pt}
\textbf{الف)}
ثابت کنید که این مکانیزم DSIC است.

\vspace*{5pt}
\textbf{ب)}
فرض کنید که خریداران صادقانه پیشنهاد می‌دهند. نشان دهید به ازای هر ثابت  $c > 0$، و مستقل از $n$، هیچ قیمت‌گذاری ناتصادفی‌ای وجود ندارد که در آن
\lr{competitive ratio}
مزایده حداقل $c$ باشد.
