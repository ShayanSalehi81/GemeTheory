در مزایده
\LR{sponsored search}
تعداد $k$ جایگاه تبلیغات وجود دارد که در کنار نتایج جست‌و‌جو به نمایش گذاشته می‌شود. در مزایده فرض می‌کنیم که جایگاه $j$ام مستقل از تبلیغ درون آن به اندازه‌ی $a_j$ احتمال کلیک دارد که این مقدار بر حسب $j$ نزولی است. بنابراین به ازای هر بار جست و جو، مطلوبیت خریدار $i$ که در جایگاه $j$ام قرار گرفته است، به اندازه‌ی
$a_j(v_i - p_j)$
خواهد بود. یکی رایج‌ترین مزایده‌ها برای سرچ اسپانسری، مزایده‌ی GSP می‌باشد که به صورت زیر تعریف می‌شود.
\begin{itemize}
    \item
    تمام خریداران را بر اساس مقداری که پیشنهاد داده‌اند مرتب می‌کنیم. (بدون کم شدن از کلیت فرض کنید
    $b_1 \geq b_2 \geq \ldots \geq b_n$
    .)

    \item
    جایگاه $i$ام را به $i$امین پیشنهاد دهنده اختصاص می‌دهیم.

    \item
    از هر پیشنهاد دهنده‌ی $i$ به اندازه‌ی $b_i + 1$ هزینه می‌گیریم.

\end{itemize}

\vspace*{5pt}
\textbf{الف)}
نشان دهید GSP به ازای هر
$k \geq 2$
و هر مقداری از نرخ‌ کلیک‌ها DSIC نیست.

\vspace*{5pt}
\textbf{ب)}
ک پروفایل bid مانند به صورت
$b_1 \geq b_2 \geq \ldots \geq b_n$
را envy-free در نظر می‌گیریم اگر برای هر خریدار $i$ و هر جایگاه $j\ne i$ داشته باشیم:
\[
    a_i(v_i - b_i + 1) \geq a_j(v_i - b_j + 1)    
\]
نشان دهید هر پروفایل envy-free یک تعادل نش نیز هست.

\vspace*{5pt}
\textbf{ج)}
نشان دهید هر تعادل envy-free در مکانیزم GSP حداقل به اندازه‌ی مکانیزم DSIC تدریس شده در درس، revenue ایجاد می‌کند.