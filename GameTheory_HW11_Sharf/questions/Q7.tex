در این سوال با مفهوم externality آشنا می‌شوید، که در ادامه‌ی درس با حالت کلی آن مواجه می‌شوید.

\vspace*{5pt}
\textbf{الف)}
یک محیط تک متغیره‌ی دلخواه با فضای تخصیص‌های شدنی $X$ را در نظر بگیرید. نشان دهید قاعده‌ی تخصیصی که رفاه جمعی را بیشینه می‌کند یعنی:
\[
x(b) = \text{argmax}_{(x_1, \ldots, x_n) \in X} \sum_{i=1}^{n} b_i x_i
\]
صعودی یکنوا است. (فرض کنید حالت‌های تساوی را با یک قاعده‌ی
\lr{Tie breaking}
رفع ابهام می‌کنیم.)


\vspace*{5pt}
\textbf{ب)}
فضای‌ حالت‌های شدنی در سوال قبل را به بردار‌های 0-1 تقلیل دهید. (فضای
$n^{[0,1]}$
) یعنی یک شرکت کننده یا می‌برد یا میبازد. قاعده‌ی پرداخت مایرسون برای قاعده‌ی تخصیص بخش قبل را بنویسید و نشان دهید اگر
$S^{*}$
مجموعه‌ی برندگان مزایده طبق قاعده‌ی تخصیص بخش قبل باشد آنگاه هر بازیکن برنده اختلاف بین یشینه‌ی رفاه جمعی شرکت کنند‌گان در حالتی که $i$ شرکت نکند و رفاه بازیکنان دیگر جز $i$ در
$S^{*}$
را پرداخت می‌کند. به این معنا هر بازیکن برنده مقداری که با مشارکتش از رفاه سایر بازیکنان کم می‌کند را می‌پردازد.