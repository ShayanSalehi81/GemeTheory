دو خودروساز
$A$
و
$B$
را در نظر بگیرید که در حال تصمیم گیری در مورد افزایش ظرفیت تولید خودشان هستند. هر یک سه انتخاب پیش روی خود دارند:
\begin{itemize}
    \item ظرفیت خود را ثابت نگه دارند. (ث)
    \item ظرفیت خود را کمی افزایش دهند (ک)
    \item ظرفیت خود را به مقدار زیاد افزایش دهند (ز)
\end{itemize}
با توجه به سود دریافتی هر یک که طبق جدول زیر است به سوالات زیر پاسخ دهید.
\LTR 
    \begin{center}
        \begin{tabular}{r | c | c | c |}
            \multicolumn{1}{r}{} &
            \multicolumn{1}{c}{ث} &
            \multicolumn{1}{c}{ک} &
            \multicolumn{1}{c}{ز} \\ \cline{2-4}
            ث       & $36,36$   & $30,40$     & $18,36$  \\ \cline{2-4}
             ک      & $40,30$   & $32,32$     & $16,24$  \\ \cline{2-4}
              ز      & $36,18$   & $24,16$     & $0,0$    \\ \cline{2-4}
        \end{tabular}
    \end{center}
\RTL

\textbf{الف)}
با فرض این که 
$A$
و
$B$
همزمان تصمیم بگیرند، فرم درختی و فرم نرمال این بازی را بنویسید و تعادل آن را به دست آورید.
\vspace{5pt}

\textbf{ب)}
در صورتی که
$A$
اول بازی کند و سپس
$B$
با دانستن انتخاب
$A$،
بازی کند، فرم درختی بازی را رسم کنید و تعادل زیربازی کامل آن را به دست آورید.
\vspace{5pt}



\textbf{ج)}
مشابه حالت ب این بار فرض کنید
$A$
انتخاب
$A$
را نمی‌داند، فرم درختی، فرم نرمال و تعادل زیربازی کامل را با این شرایط به دست آورید.