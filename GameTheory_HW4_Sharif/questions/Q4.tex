پنج راهزن در آخرین غارتشان صد سکه‌ی طلا به دست آورده‌اند و باید آن را بین خودشان تقسیم کنند. هر کدام از راهزنان می‌خواهند تعداد سکه‌هایشان بیش‌تر باشد.
همیشه رئیس تقسیم‌بندی را پیشنهاد می‌کند. در یک فرآیند دموکراتیک همه‌ی راهزنان به تقسیم‌بندی پیشنهادی رأی می‌دهند و اگر حداقل نیمی از راهزنان موافق باشند، سکه‌ها طبق پیشنهاد تقسیم می‌شوند. اگر هم رئیس نتواند موافقت حداقل نیمی از افراد را (که شامل خودش نیز می‌شود) به دست آورد، راهزنان دیگر او را سربه‌نیست می‌کنند! سپس در میان راهزنان باقی‌مانده باسابقه‌ترین رئیس می‌شود و این فرآیند تکرار می‌شود.
ترجیحات راهزنان به این صورت است:
\begin{itemize}
    \item اول از همه، هر راهزن می‌خواهد زنده بماند.
    \item سپس در صورت بقا هر راهزن می‌خواهد تعداد سکه هایش بیش‌تر باشد.
    \item در نهایت، هر راهزن ترجیح می‌دهد در صورتی که در تعداد سکه‌هایش تاثیری ندارد، دیگر راهزنان را سربه‌نیست کند.
\end{itemize}
حداکثر تعداد سکه‌هایی که رئیس اصلی می‌تواند در تمام تعادل‌های زیربازی کامل کسب کند، چقدر است؟
