آرش و بابک دو برادر هستند که در یک مسابقه‌ی تلویزیونی شرکت کرده‌اند و صد دلار برنده شده‌اند. چون آرش پیشنهاد شرکت در مسابقه را مطرح کرده‌بود وظیفه‌ی تقسیم جایزه‌شان بر عهده‌ی اوست. عملیات تقسیم جایزه به این صورت انجام می‌شود که آرش تصمیم می‌گیرد به هر کدام چه مقدار پول برسد و سپس بابک تصمیم می‌گیرد این تقسیم‌بندی را بپذیرد یا خیر. اگر بابک تقسیم‌بندی را قبول کند، به هر کدام مطابق با تقسیم‌بندی پول می‌رسد؛ اگر بابک تقسیم‌بندی را قبول نکند، پدر و مادرشان تمام پول را خرج خرید یک جاروی رباتیک می‌کنند و به بچه‌ها پولی تعلق نمی‌گیرد.
\vspace{10pt}

\textbf{الف)}
فرض کنید اگر به بابک صفر دلار پیشنهاد شود، عصبانی شده و تقسیم‌بندی را نمی‌پذیرد. تعادل زیربازی کامل را بیابید.
\vspace{5pt}

\textbf{ب)}
فرض کنید بابک آن‌قدر از جاروی رباتیک بدش می‌آید که حتی پیشنهاد صفر دلار را هم بپذیرد. تعادل زیربازی کامل را بیابید.
\vspace{5pt}

\textbf{ج)}
فرض کنید تابع سود آرش به شکل
$u_A(x_1, x_2) = x_1 - 2(x_2 - x_1)$
و تابع سود بابک به شکل
$u_B(x_1, x_2) = x_2 - 2(x_1 - x_2)$
تعریف شود که در آن
$x_1$
پول پیشنهادی برای آرش و
$x_2$
پول پیشنهادی برای بابک است. یعنی برای هر کس علاوه بر آن که پول خودش، اختلاف پولش با برادرش هم مهم است. با این فرض که هر دو نفر همچنان سود صفر را به خریدن جاروی رباتیک ترجیح می‌دهند، تعادل زیربازی کامل را بیابید.