آرش و بابک در حال مذاکره برای رسیدن به یک توافق هستند. اگر دو طرف فوراً به توافق برسند، هر کدام صد میلیون دلار سود می‌کنند. اگر دو طرف فوراً به توافق نرسند، یک ماه بعد شانس آخر را برای دستیابی به توافق خواهند داشت. اگر توافق در یک ماه حاصل شود، شرکت آرش ۴۰ میلیون دلار و بابک ۵۵ میلیون دلار سود می‌کنند.
مذاکره به روش زیر عمل می کند: در دور اول آرش برای امضای قرارداد، مبلغی را از بابک می‌خواهد. اگر بابک بپذیرد، توافق حاصل می‌شود و سود آرش صد میلیون دلار به اضافه‌ی پرداختی بابک است. سود بابک هم صد میلیون دلار منهای مبلغی است که متعهد شده‌است پرداخت کند.
اگر بابک پیشنهاد را رد کند، در دور دوم مذاکرات مبلغی را به آرش پیشنهاد می‌کند که آن را برای امضای قرارداد دریافت می‌کند. این بار هم سود هر دو نفر به‌طور مشابهی تحت تاثیر مبلغ پرداختی قرار می‌گیرد.
\vspace{10pt}

\textbf{الف)}
در تعادل زیربازی کامل آرش چه مبلغی را از بابک درخواست می‌کند؟
\vspace{5pt}

\textbf{ب)}
اکنون فرض کنید ابتدا بابک و سپس آرش پیشنهادشان را ارائه می‌کنند. در تعادل زیربازی کامل بابک چه مبلغی را از آرش درخواست می‌کند؟
