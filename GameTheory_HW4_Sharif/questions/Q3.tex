یک خودکار دو دلاری به حراج گذاشته شده است. آرش و بابک که هر کدام سه دلار پول دارند، به دنبال به دست آوردن این خودکار هستند. پیشنهاد دادن از آرش شروع می‌شود و دو نفر به نوبت پیشنهاد خود را ارائه می‌کنند. هر پیشنهاد می‌تواند یک عدد صحیح مثبت باشد که از آخرین پیشنهاد بیش‌تر است. در هر نوبت هر کس می‌تواند یک پیشنهاد معتبر ارائه کند و یا از ادامه‌ی حراج انصراف بدهد. در صورت انصراف خودکار به شرکت‌کننده‌ی دیگر می‌رسد. هر دو نفر باید آخرین پیشنهادشان (اگر پیشنهادی داده‌باشند) را پرداخت کنند و هیچ‌کس نمی‌تواند پیشنهادی بیش‌تر از ثروتش (سه دلار) ارائه کند. حراج را به شکل یک بازی گسترده مدل کنید و تعادل زیربازی کامل آن را پیدا کنید.
