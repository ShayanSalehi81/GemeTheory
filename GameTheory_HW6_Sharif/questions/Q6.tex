فرض کنید
$x$
یک متغیر تصادفی باشد که یکی از مقادیر
$a_1, \ldots, a_n$
را اختیار می‌کند. می‌دانیم
$x$
از یکی از توزیع‌های
$p$
یا 
$q$
آمده است. یک نمونه تصادفی از
$x$
مشاهده می‌کنیم. هدف تعیین این است که
$x$
از توزیع‌
$p$
آمده است یا
$q$.

فرض کنید
$0 \leq x_j \leq 1$.
می‌خواهیم
$x_j$ها
را طوری تعیین کنیم که:
\vspace*{5pt}

\hspace*{5pt}
۱.  اگر نمونه مشاهده شده
$a_j$
باشد، ما به احتمال
$x_j$
توزیع‌
$p$
را انتخاب می‌کنیم.
\vspace*{5pt}

\hspace*{5pt}
۲.  اگر توزیع
$q$
باشد، احتمال انتخاب
$p$
حداکثر
$0.01$
باشد.
\vspace*{5pt}

\hspace*{5pt}
۳.  اگر توزیع 
$p$
باشد، احتمال انتخاب
$p$
بیشینه باشد.
\vspace*{5pt}

(به شرط این که اگر توزیع
$q$
بود احتمال انتخاب
$p$
حداکثر
$0.01$
باشد، کاری کنیم که اگر توزیع
$p$
باشد، احتمال انتخاب
$p$
بیشنیه شود.)

یک برنامه خطی برای پیدا کردن
$x_j$ها
بنویسید.