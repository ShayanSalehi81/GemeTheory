آبولف و بهزاد می‌خواهند با یکدیگر بازی کنند. در این بازی آبولف $N$ و بهزاد $M$ کنش متفاوت دارند. اگر آبولف کنش $i$ام و بهزاد کنش $j$ام را بازی کند، سود آبولف
$u^{(1)}_{i,j}$
و سود بهزاد
$u^{(2)}_{i,j}$
خواهد بود.

به شما $n$، $m$ و دو ماتریس 
$u^{(1)}$
و 
$u^{(2)}$ 
داده می‌شود، یک تعادل نش ترکیبی برای این بازی پیدا کنید.

\subsection*{ورودی}
در خط اول ورودی دو عدد صحیح $N$ و $M$ 
($1 \leq N, M \leq 7$) 
می‌آیند که به‌ترتیب تعداد کنش‌های آبولف و تعداد کنش‌های بهزاد هستند. 

در خط $i$ام از $N$ خط بعدی، $M$ عدد صحیح 
$u^{(1)}_{i,1}, u^{(1)}_{i,2}, \cdots, u^{(1)}_{i,m} \,$
($-50 \leq u^{(1)}_{i,j} \leq 50 \,$)
می‌آیند که ماتریس سود مربوط به آبولف را مشخص می‌کند.

در خط $i$ام از $N$ خط بعدی، $M$ عدد صحیح 
$u^{(2)}_{i,1}, u^{(2)}_{i,2}, \cdots, u^{(2)}_{i,m} \,$
($-50 \leq u^{(2)}_{i,j} \leq 50 \,$)
می‌آیند که ماتریس سود مربوط به بهزاد را مشخص می‌کند.

\subsection*{خروجی}
در خط اول $N$ عدد اعشاری
$p_1, p_2, \cdots, p_N$
را با دقت \textbf{۶ رقم} اعشار چاپ کنید که استراتژی آبولف را نشان می‌دهد.         

در خط دوم $M$ عدد اعشاری
$q_1, q_2, \cdots, q_M$
را با دقت \textbf{۶ رقم} اعشار چاپ کنید که استراتژی بهزاد را نشان می‌دهد.

\textbf{دقت کنید در صورتی که بازی چند تعادل نش داشته باشد، کافی است یکی را به دلخواه چاپ کنید.}

پاسخ شما روی هر تست پذیرفته می‌شود اگر تمامی شروط پایین برقرار باشد.
\begin{itemize}
\item
فرض کنید
$P = \sum_{i=1}^{N} p_i$. 
باید
$\mid P - 1 \mid \leq 10^{-3}$
برقرار باشد.
\item
فرض کنید
$Q = \sum_{i=1}^{M} q_i$.
باید
$\mid Q - 1 \mid \leq 10^{-3}$
برقرار باشد.
\item
به ازای هر
$1 \leq i \neq j \leq n$
که
$p_i, p_j \geq 10^{-6}$
است، نامساوی زیر برقرار باشد.
$$\mid \sum_{k=1}^{M} q_k \cdot u^{(1)}_{i,k} - \sum_{k=1}^{M} q_k \cdot u^{(1)}_{j,k} \mid < 10^{-3}$$
\item
به ازای هر
$1 \leq i \neq j \leq n$
که
$p_i \geq 10^{-6}$
و
$p_j < 10^{-6}$
است، نامساوی زیر برقرار باشد.
$$\mid \sum_{k=1}^{M} q_k \cdot u^{(1)}_{i,k} - \sum_{k=1}^{M} q_k \cdot u^{(1)}_{j,k} \mid > -10^{-3}$$
\item
دو شرط بالا برای استراتژی بهزاد نیز به طور مشابه بررسی می‌شود.
\end{itemize}
\pagebreak
\subsection*{مثال}
\subsubsection*{ورودی نمونه ۱}
\LTR
\begin{latin}
\begin{examplebox}
    3 3 \\
    0 -1 1 \\
    1 0 -1 \\
    -1 1 0 \\
    0 1 -1 \\ 
    -1 0 1 \\
    1 -1 0 
\end{examplebox}
\end{latin}
\RTL
\subsubsection*{خروجی نمونه ۱}
\LTR
\begin{latin}
\begin{examplebox}
    0.333333 0.333333 0.333333 \\
    0.333333 0.333333 0.333333
\end{examplebox}
\end{latin}
\RTL
\subsubsection*{ورودی نمونه ۲}
\LTR
\begin{latin}
\begin{examplebox}
    2  2 \\
   -2  0 \\
   -10 2 \\
   -8 -2 \\
    8  2
\end{examplebox}
\end{latin}
\RTL
\subsubsection*{خروجی نمونه ۲}
\LTR
\begin{latin}
\begin{examplebox}
    0.500000 0.500000 \\
    0.200000 0.800000
\end{examplebox}
\end{latin}
\RTL
\subsubsection*{ورودی نمونه ۳}
\LTR
\begin{latin}
\begin{examplebox}
    3 3 \\
    9 -9 1 \\
    7 5 1 \\
    9 -10 -7 \\
    7 6 -10 \\
    -1 3 -8 \\
    -8 3 6
\end{examplebox}
\end{latin}
\RTL
\subsubsection*{خروجی نمونه ۳}
\LTR
\begin{latin}
\begin{examplebox}
    1.000000 0.000000 0.000000 \\
    1.000000 0.000000 0.000000
\end{examplebox}
\end{latin}
\RTL
\textbf{توضیحات نمونه ۳:} همه تعادل‌های نش این بازی شامل موارد زیر است که هر کدام را چاپ کنید قابل قبول است.
\begin{itemize}
\item
آبولف با استراتژی
$(1, 0, 0)$
و بهزاد با استراتژی
$(1, 0, 0)$
بازی کند.
\item
آبولف با استراتژی
$(0, 1, 0)$
و بهزاد با استراتژی
$(0, 1, 0)$
بازی کند.
\item
آبولف با استراتژی
$(0.8, 0.2, 0)$
و بهزاد با استراتژی
$(0.875, 0.125, 0)$
بازی کند.
\end{itemize}
\subsubsection*{ورودی نمونه ۴}
\LTR
\begin{latin}
\begin{examplebox}
    3 2 \\
    8 -5 \\
    -3 4 \\
    -5 -4 \\
    8 9 \\
    1 -2 \\
    8 5
\end{examplebox}
\end{latin}
\RTL
\subsubsection*{خروجی نمونه ۴}
\LTR
\begin{latin}
\begin{examplebox}
    0.750000 0.250000 0.000000 \\ 
    0.450000 0.550000
\end{examplebox}
\end{latin}
\RTL