در این سوال با حالتی دیگر از تعادل به نام تعادل همبسته یا همان
\LR{correlated equilibrium}
آشنا می‌شوید که مفهوم کلی‌تری نسبت به تعادل نش است. در ادامه در قالب یک مثال این تعادل توضیح داده می‌شود.

بازی نبرد جنسیت‌ها را به خاطر بیاورید. جدول سودمندی و احتمال انتخاب هر کنش در ادامه آمده است. عدد رو به روی هر کنش بیانگر احتمال آن کنش در استراتژی ترکیبی آن بازیکن است.
\begin{center}
\begin{tabularx}{0.7\textwidth}{|Y|Y|Y|}
    \hline
    خرید کردن (۱/۳) & تماشای فیلم (۲/۳) &  \\
    \hline
    $(0, 0)$ & $(1,2)$ & تماشای فیلم (۱/۳) \\
    \hline
    $(0, 0)$ & $(1,2)$ & خرید کردن (۲/۳) \\
    \hline
\end{tabularx}
\end{center}
حال فرض کنید پیش از بازی از یک کیف حاوی دو گوی با برچسب‌های «خرید» و «فیلم» یک گوی به تصادف انتخاب می‌شود و دو بازیکن توافق می‌کنند که استراتژی خود را بر اساس گوی بیرون آمده، انتخاب کنند. در این حالت می‌توان دید که هیچ یک از بازیکنان تمایلی به انتخاب استراتژی متفاوت نسبت به گوی بیرون آمده ندارند؛ زیرا متوسط سودمندی در این حالت برای هر یک از آنها افزایش پیدا کرده است.

\subsection*{به نفع همه}
تعادل همبسته در یک بازی می‌تواند بر افزایش رفاه اجتماعی و حتی سودمندی متوسط هر یک از بازیکنان تاثیر مثبتی بگذارد. 
در حالت قبلی بازی نبرد جنسیت‌ها، سودمندی متوسط بازیکنان به شکل زیر محاسبه می‌شد.

$$E[pay\,\,off] = \frac{2}{3} \cdot (\frac{1}{3} \cdot 2 + \frac{2}{3} \cdot 0) + \frac{1}{3} \cdot (\frac{2}{3} \cdot 1 + \frac{1}{3} \cdot 0) = \frac{6}{9}$$

حال اگر پیش از انجام بازی، از داخل کیف یک گوی به تصادف انتخاب شود؛ سودمندی مورد انتظار بازیکنان افزایش پیدا خواهد کرد. در واقع این قرعه‌کشی کاری می‌کند که حالت‌های بد هیچ‌گاه اتفاق نیفتد. در این حالت متوسط سود هر بازیکن به شکل زیر محاسبه می‌شود.

$$E[pay\,\,off] = \frac{1}{2} \cdot 1 + \frac{1}{2} \cdot 2 = \frac{3}{2}$$

در حالت کلی می‌توان فرض کرد که یک فرد خارج از بازی به نام ناظر هر یک از استراتژی‌های ممکن را با احتمال مشخصی انتخاب کرده و به بازیکنان کنش مربوط به آنها را در قالب یک سیگنال اعلام می‌کند.

\subsection*{تعریف رسمی}
یک بازی با بازیکنان $p = 1,2,...,n$ را در نظر بگیرید. مجموعه استراتژی هر بازیکن با $S_p$ نمایش داده می‌شود. بر همین اساس استراتژی پروفایل $S$ به شکل زیر تعریف می‌گردد.

$$ S = \prod_{p=1}^{n} S_p $$

می‌دانیم که $S_{-q}$ بیانگر استراتژی پروفایل همه بازیکنان به جز بازیکن $q$ ام است. همچنین سود هر بازیکن به شکل تابع $u^p$ بر روی استراتژی پروفایل $S$ تعریف می‌شود.

حال توزیع احتمالاتی $x$ بر روی $S$ را در نظر بگیرید. به ازای هر $\overline{s} \in S_{-p}$ احتمال انتخاب شدن استراتژی $i$ توسط بازیکن $p$ در حالی که سایر بازیکنان با استراتژی پروفایل $\overline{s}$ بازی کنند؛ با $x_{i,\overline{s}}$ نمایش داده می‌شود. به شکل مشابه $u_{i,\overline{s}}^{p}$ سودمندی بازیکن p به ازای انتخاب استراتژی $i \in S_p$ را نشان می‌دهد؛ در حالی که سایر بازیکنان استراتژی پروفایل $\overline{s}$ را داشته باشند.
توزیع $x$ یک تعادل همبسته نامیده می‌شود اگر بازیکن $p$ استراتژی پیشنهادی $i$ را بپذیرد.

$$\sum_{\overline{s} \in S_{-p}}{u_{i,\overline{s}}^{p}x_{i, \overline{s}}} \geq \sum_{\overline{s} \in S_{-p}}{u_{j,\overline{s}}^{p}x_{i, \overline{s}}} \; \forall{p} \: and \: \forall{i,j} \in S_p$$

در واقع سودمندی مورد انتظار ناشی از انتخاب استراتژی پیشنهاد شده نباید کمتر از بازی با سایر استراتژی‌ها باشد.

\subsection*{به نفع همه؛ برای بعضی‌ها بیشتر}
چنانچه گفته شد به کمک ناظر می‌توان سود اجتماعی را افزایش داد. اما ممکن است از نظر ناظر، یک واحد سود یک بازیکن به اندازه چند واحد سود بازیکن دیگر برای جامعه مفید باشد. بنابراین متوسط سود او در محاسبه رفاه اجتماعی با یک ضریب همراه خواهد بود.

فرض کنید به بازیکن اول و دوم به ترتیب ضریب ۱ و ۲ اختصاص داده شده است. حال اگر متوسط سود این دو بازیکن به ترتیب ۱۰ و ۲۰ باشد؛ رفاه اجتماعی وزن دار برابر ۵۰ خواهد بود.

\subsection*{به یک ناظر نیازمندیم}
در این سوال از شما به عنوان یک ناظر دعوت شده است که در یک بازی دو نفره شرکت کنید. شما باید بر اساس جدول سودمندی بازیکنان، احتمال انتخاب هر یک از خانه‌های جدول را مشخص کنید به شکلی که…
\begin{enumerate}
\item
بازی در حالت تعادل همبسته قرار داشته باشد تا بازیکنان تمایلی به سرکشی از سیگنال‌های شما نداشته باشند.
\item
سود اجتماعی بیشینه شود. از آنجا که شما ممکن است با یک بازیکن بیشتر حال کنید؛ می‌تواند سودمندی شخصی او را با اعمال یک ضریب در سودمندی اجتماعی تاثیر دهید.
\end{enumerate}

\subsection*{ورودی}
در اولین سطر به شما دو عدد گویای $0 \leq N, M \leq 20$ با یک فاصله داده می‌شود که به ترتیب میزان علاقمندی شما به هر یک از دو بازیکن است.

در سطر بعدی به شما دو عدد طبیعی $0 \leq X, Y \leq 20$ با یک فاصله داده می‌شود که به ترتیب تعداد کنش‌های هر یک از بازیکنان است.

در ادامه جدول سودمندی بازیکنان به شما داده می‌شود. از آنجا که این جدول $X$ سطر و $Y$ ستون دارد؛ مقادیر آن در $X$ سطر ظاهر می‌شود. در هر سطر نیز $2Y$ عدد صحیح $-50 \leq a_i \leq 50$ می‌آید تا هر خانه از جدول به شکل دو عدد متوالی نمایش داده شود. به ورودی نمونه زیر توجه کنید.
\LTR
\begin{latin}
\begin{examplebox}
    2.3 1.1 \\
    2 2 \\
    1 2 3 4 \\
    5 6 7 8
\end{examplebox}
\end{latin}
\RTL
در این ورودی جدول سودمندی به شکل زیر است.
\begin{center}
    \begin{tabularx}{0.7\textwidth}{|Y|Y|Y|}
        \hline
        کنش دوم & کنش اول &  \\
        \hline
        $(3,4)$ & $(1,2)$ & کنش اول \\
        \hline
        $(7,8)$ & $(5,6)$ & کنش دوم \\
        \hline
    \end{tabularx}
    \end{center}

\subsection*{خروجی}
در اولین خط خروجی متوسط رفاه اجتماعی وزن دار چاپ می‌گردد. 
سپس باید در $X$ خط بعدی خروجی، احتمال انتخاب هر خانه از جدول توسط ناظر به شکل جدولی نشان داده شود. همه اعداد باید تا ۶ رقم اعشار دقت داشته باشند.

\subsection*{زبان برنامه‌نویسی}
در این سوال زبان مورد استفاده python خواهد بود و همچنین ماژول simplex.py برای حل برنامه‌نویسی خطی در اختیار شما قرار می‌گیرد.

\subsubsection*{ورودی نمونه ۱}
\LTR
\begin{latin}
\begin{examplebox}
    1 1 \\
    2 2 \\
    0 0 7 2 \\
    2 7 6 6
\end{examplebox}
\end{latin}
\RTL
\subsubsection*{خروجی نمونه ۱}
\LTR
\begin{latin}
\begin{examplebox}
    10.500000 \\
    0.000000 0.250000 \\
    0.250000 0.500000
\end{examplebox}
\end{latin}
\RTL
\subsubsection*{ورودی نمونه ۲}
\LTR
\begin{latin}
\begin{examplebox}
    1 1 \\
    2 2 \\
    0 0 4 1 \\
    1 4 3 3
\end{examplebox}
\end{latin}
\RTL
\subsubsection*{خروجی نمونه ۲}
\LTR
\begin{latin}
\begin{examplebox}
    5.333333 \\
    0.000000 0.333333 \\
    0.333333 0.333333
\end{examplebox}
\end{latin}
\RTL
\subsubsection*{ورودی نمونه ۳}
\LTR
\begin{latin}
\begin{examplebox}
    1 1 \\
    3 3 \\
    6 6 -2 0 0 7 \\
    2 2 2 2 0 0 \\
    0 0 0 0 3 3
\end{examplebox}
\end{latin}
\RTL
\subsubsection*{خروجی نمونه ۳}
\LTR
\begin{latin}
\begin{examplebox}
    8.000000 \\
    0.500000 0.000000 0.000000 \\
    0.250000 0.250000 0.000000 \\
    0.000000 0.000000 0.000000
\end{examplebox}
\end{latin}
\RTL