می‌خواهیم فضاپیمایی را از نقطه
$A$
به نقطه
$B$
بفرستیم. فرض کنید زمان
$t$
بر حسب ثانیه و همچنین
$a_t$،
$v_t$
و
$x_t$
به ترتیب مکان، سرعت و شتاپ فضاپیما در لحظه
$t$
باشد. با در نظر گرفتن زمان به صورت گسسته و به صورت تقریبی، معادله‌های زیر برقرار است.
\begin{align*}
    x_{t + 1} &= x_t + v_t \\
    v_{t + 1} &= v_t + a_t
\end{align*}
فرض کنید که مقدار شتاب
$a_t$
در هر لحظه توسط ما کنترل می‌شود و اندازه آن
$(|a_t|)$
ضریبی ثابت از مقدار سوخت استفاده شده در ثانیه
$t$
تا 
$t + 1$
باشد.

می‌خواهیم فضاپیما از مکان
$x_0 = 0$
و با سرعت
$v_0 = 0$
از زمین بلند شده، و همچنین
$T$
ثانیه در ارتفاع
$x_T = d$
به سرعت 
$v_T = 0$
برسد.
\vspace*{10pt}

\textbf{الف)}
می‌خواهیم کل سوخت مصرف‌شده در هنگام سفر یعنی
$\sum_{t=0}^{T} |a_t|$
را مینیموم کنیم.

\vspace*{5pt}
\textbf{ب)}
می‌خواهیم بیشترین مقدار سوخت مورد نیاز در هر لحظه
$t$
یعنی
$\max_{t=0}^{T} \{|a_t|\}$
را مینیموم کنیم.

\vspace*{5pt}
در هر حالت مسئله را به صورت یک برنامه‌ریزی خطی مدل کنید.
