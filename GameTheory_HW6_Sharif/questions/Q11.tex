برای گراف جهت‌دار
$f\,:\,A \to R, D = (V,A)$
را یک جریان دوری می‌گوییم هرگاه:
\[
    \sum_{a \in \delta^{in}(v)} f(a) =  \sum_{a \in \delta^{out}(v)} f(a) \quad \quad \forall v \in V
\]
\vspace*{5pt}
\textbf{الف)}
گراف جهت‌دار
$D = (V, A)$
را در نظر بگیرید. نشان دهید ماتریس برخورد
$D$
تماما تک‌پیمانه‌ای است.

\vspace*{5pt}
\textbf{ب)}
هر بردار
$x \in R^{A}$
را می‌توان تابعی روی یال‌های
$D$
در نظر گرفت. اگر
$M$
ماتریس برخورد گراف بالا باشد، نشان دهید
$M x = 0$
اگر و تنها اگر
$x$
یک جریان دوری روی یال‌های
$D$
باشد.

\vspace*{5pt}
\textbf{ج)}
فرض کنید
$D = (V, A)$
گراف جهت‌دار به همان صورت بالا بوده و 
$c\,:\,A \to Z$
و
$d\,:\,A \to Z$
دو بردار در
$Z^A$
باشند. نشان دهید اگر جریان دوری مانند
$x$
روی
$A$
وجود داشته باشد به طوری که 
$c \leq x \leq d$
آنگاه جریان دوری صحیحی مانند
$z$
روی
$A$
وجود خواهد داشت به طوری که
$c \leq z \leq d$.

\vspace*{5pt}
\textbf{ج)}
فرض کنید
$D = (V, A)$
گراف جهت‌دار به همان صورت بالا بوده و 
$c,d\,:\,A \to Z$
به طوری که
$d \leq c$. 
نشان دهید جریان دوری مانند
$f$
وجود دارد به طوری که 
$d \leq f \leq c$
اگر و تنها اگر برای هر زیرمجموعه
$U \in V$ 
داشته باشیم:
\[
    \sum_{a \in \delta^{in}(U)} d(a) \leq \sum_{a \in \delta^{out}(U)} c(a)  
\]